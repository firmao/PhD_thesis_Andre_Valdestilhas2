\section{Research Questions and Contributions}\label{researchquestions}
%For each motivation, we formulate a research question (RQ) and state our contribution towards it.
For each motivation, we formulate a research question (RQ) towards the contribution of this thesis, Identifying, Relating, Consisting, and Querying Large Heterogeneous RDF Sources.

\begin{itemize}
    \item (RQ1) How to Identify Datasets in Large Heterogeneous RDF Sources? %WIMU
    \begin{itemize}
        \item Covered in the chapter \ref{ch:wimu}.
    \end{itemize}
    \item (RQ2) How to create relations among a Large Amount of RDF Datasets? %RELOD
    \begin{itemize}
        \item Covered in the chapter \ref{ch:lodindex}.
    \end{itemize}
    \item (RQ3) How to obtain Similar Resources Using String Similarity? %MFKC
    \begin{itemize}
        \item Covered in the chapter \ref{ch:mfkc}.
    \end{itemize}
    \item (RQ4) How to tackle Heterogeneity in DBpedia Identifiers? %DBpediaSameAs
    \begin{itemize}
        \item Covered in the chapter \ref{ch:dbpediasameas}.
    \end{itemize}
    \item (RQ5) How to Detect Erroneous Links in Large-Scale RDF Datasets? %CEDAL
    \begin{itemize}
        \item Covered in the chapter \ref{ch:cedal}.
    \end{itemize}
    \item (RQ6) How to Query Large Heterogeneous RDF Datasets? %WIMUQ
    \begin{itemize}
        \item Covered in the chapter \ref{ch:wimuq}.
    \end{itemize}
\end{itemize}

% In order to facilitate the reading of the thesis, here we present a list of the research questions organized by chapter.
% \begin{itemize}
%     \item Thesis question: Given a large amount of heterogeneous RDF sources, how to identify, relate, consist and query those RDF sources?
%     \item Chapter \ref{ch:relating}
%     \begin{itemize}
%         \item How to identify and relate large amount of RDF sources?
%         \begin{itemize}
%             \item Given RDF sources $D_s, D_t$ and relation $R$
%             \item Find $M = \{(s,t) \in D_s \times D_t: R(s, t)\}$
%         \end{itemize}
%         \item How to find a better way to compute $M$ directly?
%         \item How to improve the na\"ive computation of M' $\in O(n^2)$
%         \item Is there a way to know in which dataset a given URI was defined? 
%         \item How to identify and quantify similar datasets for a given Dataset?
%         \item Could the results from different data sets complement each other?
%         \item How the detection of duplicated and chunk datasets can help in the process of matching a large amount of datasets?
%     \end{itemize}
%     \item Chapter \ref{ch:consistency}
%     \begin{itemize}
%         \item How to tackle heterogeneity in DBpedia identifiers?
%         \item How to tackle the inconsistency in Linksets?
%         \item Is there a time-efficient algorithm to detect erroneous links in large-scale link repositories?
%         \item Is there an approach to discover whether a linkset is consistent without computing all closures required by the property axiom?
%     \end{itemize}
%     \item Chapter \ref{ch:wimuq}
%     \begin{itemize}
%         \item Is it possible to Identify automatically relevant sources from heterogeneous RDF data, even with non-dereferenceable URIs, improving the resultset retrieval?
%         \item How many datasets are most likely to execute a given SPARQL query?
%         \item How ReLOD increase the number of datasets identified by wimuQ?
%     \end{itemize}
% \end{itemize}
